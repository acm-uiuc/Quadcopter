\documentclass{acmsiggraph}

\usepackage[scaled=.92]{helvet}
\usepackage{times}
\usepackage{graphicx}
\usepackage{parskip}
\usepackage{fixltx2e}
\usepackage[labelfont=bf,textfont=it]{caption}

\TOGonlineid{0}

\title{Placeholder}

\author{Nikoli Dryden \thanks{dryden2@illinois.edu} %
\and Bryan Plummer \thanks{bplumme2@illinois.edu}}
\pdfauthor{Nikoli Dryden}

\begin{document}

\maketitle

\begin{abstract}
An abstract.
\end{abstract}

\section{Introduction}
Micro aerial vehicles (MAV's) which include quadrotor and unmanned aerial vehicles (UAV's) have the potential to become more 
prominent over the next several years.  These aircraft allow easy access for many places which are difficult for humans 
to go in a safer manner for both humans and the environment and have applications reaching into search and rescue, tracking,
mapping, and others.  In order for the higher level tasks to be handled these vehicles must have a reliable 
navigation system.

There are GPS solutions for navigation work out of the box (e.g. the AscTec series of quadrotors), but are limited in use
to locations where GPS is available.  While much progress has been made to solve to create a navigation solution
for these cases, we still strive to have a system that works in unknown, GPS denied, and cluttered environments.  As many 
MAV's are limited in both carrying capacity and power, laser range finders are too heavy and consume too much power to be of 
use. Stereo camera's become highly inaccurate after a set distance rendering them of no more use than using a single camera.

The common structure from motion approach to visual navigation using a single camera has been shown to suffer from some 
drawbacks.  In order to generate a 3D map, two types of camera translation may be necessary~\cite{shah2010}.  While
a quadrotor may be capable of such a maneuver, a UAV is not.  In~\cite{shah2009} the amount of computation required using this 
approach increased by about 15 times with only an increase from 8 to 35 feature points, and the number of features in a real
scene can number in the hundreds or thousands.

As an attempt to solve this problem~\cite{lee2011} proposed an approach to reduce the number of points required to 
represent the 3D structure of a scene.  The authors used multiscale oriented patches (MOPS)~\cite{BSW05} to create outlines of 
objects.  Since the outlines themselves are unable to tell if an outline is empty or not, they used the 3D location of SIFT 
features~\cite{lowe2004} located within the outlines to obtain this information.  

Although the proposed method cited collision avoidance for a quadrotor as its intended application, the authors did not 
dispense any data on its effectiveness in real scenes from images taken using a quadrotor.  This paper provides just such an 
evaluation using images taken from the AR-Drone 2.0 quadrotor with the processing being performed on a laptop connecting 
to the quadrotor using a wireless network. In addition, we evaluate the use of the parallel tracking and mapping (PTAM)
~\cite{klein07parallel} approach that was adapted as an online solution for quadrotors in~\cite{weiss2011}.

\section{Related Work}
Due to weight and energy restrictions aboard MAV's a vision based approach is seen as one of the most viable navigation 
solutions in GPS denied environments.  Progress in using visual senors for navigation purposes include performing specific 
tasks such as in~\cite{johnson2005} where a single camera was used to guide and land a MAV, or~\cite{huang2011isrr} where an 
RGB-D camera was used to map an environment and localize a quadrotor within that map but is restricted for use in indoor
environments.

Another common approach in the literature uses optical flow to estimate the relative motion between the frames of a camera.
In these systems one is able to detect collisions by measuring the relative rate of expansion of objects.  This method 
behaves poorly when light intensity doesn't remain constant~\cite{horn1981} and tends to be sensitive to noise and
an unstable camera.  As all MAV's will contribute to some vibrations in the camera, this leads to inaccurate estimates in
flow vectors.

Some recent work attempts to use image segmentation to identify an obstacle and build a dense map around it~\cite{ha2012}.  
This approach shares some motivations as the first method attempted in this work in that it attempts to reduce the number
of computations required for navigation by using a limited number of feature points.  As with~\cite{lee2011}, the approach
has only be evaluated in theory.  The authors note that in some cases image segmentation is not possible and intend to 
evaluate and extend their work to ascertain any benefits it may have.

\section{Approach}


\section{Experiments}


\section{Analysis}


\bibliographystyle{acmsiggraph}
\bibliography{bib_all}
\end{document}


