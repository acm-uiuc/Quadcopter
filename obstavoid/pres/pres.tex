\documentclass[xcolor=x11names,compress,t]{beamer}

%% Beamer Packages %%%%%%%%%%%%%%%%%%%%%%%%%%%%%%%%
\usepackage{graphicx}

%% Beamer Layout %%%%%%%%%%%%%%%%%%%%%%%%%%%%%%%%%%
%\useoutertheme[subsection=false,shadow]{miniframes}
\useinnertheme{default}
\usefonttheme{serif}
\usepackage{cancel}
\usepackage{palatino}

\setbeamerfont{title like}{shape=\scshape}
%\setbeamerfont{frametitle}{shape=\scshape}

\definecolor{ncsablue}{RGB}{12,81,156}

\setbeamercolor*{lower separation line head}{bg=gray}
\setbeamercolor*{normal text}{fg=black,bg=white}
\setbeamercolor*{alerted text}{fg=red}
\setbeamercolor*{example text}{fg=black}
\setbeamercolor*{structure}{fg=black}
\setbeamercolor*{title}{fg=black}
\setbeamercolor*{titlelike}{fg=ncsablue}
 
\setbeamercolor*{palette tertiary}{fg=black,bg=black!10}
\setbeamercolor*{palette quaternary}{fg=black,bg=black!10}

\beamertemplatenavigationsymbolsempty
\setbeamertemplate{footline}[page number]

\renewcommand{\(}{\begin{columns}}
\renewcommand{\)}{\end{columns}}
\newcommand{\<}[1]{\begin{column}{#1}}
\renewcommand{\>}{\end{column}}
%%%%%%%%%%%%%%%%%%%%%%%%%%%%%%%%%%%%%%%%%%%%%%%%%%

\begin{document}

\abovedisplayskip=6pt

%%%%%%%%%%%%%%%%%%%%%%%%%%%%%%%%%%%%%%%%%%%%%%%%%%%%%%
%%%%%%%%%%%%%%%%%%%%%%%%%%%%%%%%%%%%%%%%%%%%%%%%%%%%%%
%
% FRONT MATTER AND INTRODUCTION
%

\begin{frame}
\title{Obstacle Avoidance for a Quadrotor}
%\subtitle{ }
\author{Nikoli Dryden \and Bryan Plummer}
\date{}
\titlepage
\end{frame}

%%%%%%%%%%%%%%%%%%%%%%%%%%%%%%%%%%%%%%%%%%%%%%%%%%%%%%

\begin{frame}
  \frametitle{Introduction}
  \begin{itemize}
  \item Obstacle avoidance for an ARDrone 2.0 quadrotor
  \item Use the approach in \emph{Obstacle avoidance for small UAVs using monocular vision} (Lee et al. 2011)
  \item Test with real data as opposed to simulations
  \end{itemize}
\end{frame}

\begin{frame}
  \frametitle{Proposed Solution}
  \begin{itemize}
  \item Extract SIFT and MOPS features from two images
  \item Match SIFT and MOPS features in images to locate points in 3D
  \item Use MOPS to get object outlines
  \item Use SIFT to get internal object structure
  \item Determine type of object
  \end{itemize}
\end{frame}

\begin{frame}
  \frametitle{MOPS}
  \begin{itemize}
  \item Multi-Scale Oriented Patches (\emph{Image Matching using Multi-Scale Oriented Patches} (Brown et al. 2004))
  \item TODO: Include sample image
  \end{itemize}
\end{frame}

\begin{frame}
  \frametitle{Some Results: SIFT}
  Example of SIFT matches on data
\end{frame}

\begin{frame}
  \frametitle{Some Results: MOPS}
  Example of MOPS matches on data
\end{frame}

\begin{frame}
  \frametitle{Some Results}
  Discuss how MOPS doesn't work well and the paper isn't robust
\end{frame}

\begin{frame}
  \frametitle{A Better Idea}
  Discuss idea of using SIFT features on contours
\end{frame}

\begin{frame}
  \frametitle{Further Results}
  
\end{frame}

\begin{frame}
  \frametitle{Conclusions}
  
\end{frame}

%%%%%%%%%%%%%%%%%%%%%%%%%%%%%%%%%%%%%%%%%%%%%%%%%%%%%%

\end{document}